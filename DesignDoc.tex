\documentclass{report}
\usepackage{graphicx}
\author{Patrick Lindsay}
\title{Building a One-Dimensional Simulation of a Rocket: Design}

\begin{document}
\maketitle
\tableofcontents

\section{Data Design}
	The simulation necessitates only trivial data storage.  I chose to use arrays to store values for height and velocity. The index represents the time across the rocket's flight.
\section{Architecture Design}
	The simulation takes the input from the input panel and sends it to the server with a PHP POST method. The server executes an algorithm based on this data.  The algorithm takes the rocket's initial values and runs through the equations for Newton's Laws of Motion, logging the results into arrays at a sample rate of once per second.  It does this until the rocket's height becomes less than or equal to zero.  At that point, arrays are sent to the output panel where the simulation's results are displayed.
\section{Interface Design}
	The interface will comprise an input panel and an output panel. The input panel will have two text inputs and two buttons. The text inputs will be for Initial Height and Initial Velocity.  One button will run the simulation and take the user to the output panel.  The other button will reset the form. \\
	The output panel will have several components.  It will have a field for displaying all of the data generated by the simulation in CSV form.  It will also have a graph, powered by JavaScript, which will display line graphs of the height and velocity vs. time.  The user will be able to toggle these lines with check boxes or radio buttons.  The output panel will also have two buttons.  One for downloading the CSV data and one for going back to the input panel.
\section{Procedural Design}
	The simulation and its website will adhere to standard coding conventions.  All code will be verified by the W3C Validator. 
\end{document}